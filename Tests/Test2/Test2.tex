\documentclass[12pt]{extarticle} 
\usepackage[T1,T2A]{fontenc}
\usepackage[utf8]{inputenc}
\usepackage[top=2.5cm,bottom=2.5cm,left=2.5cm,right=2.5cm]{geometry}
\usepackage[english,ukrainian]{babel}
\usepackage{indentfirst} % indents first par

\usepackage{graphicx}
\graphicspath{ {./images/} } % for inserting images
\usepackage{float} %\begin{figure}[H] to force image placement
\usepackage{amsfonts,amsmath,amsthm,amscd,amssymb,latexsym, mathtools}

\title{Data Analysis \\  Тест до лекції 2}
\author{Бойченко Вікторія}
\date{}

\begin{document}

\maketitle

\begin{enumerate}

\item Які цілі у описової статистики, а які у вивідної?

Описова статистика цікавиться виключно властивостями спостережуваних даних, коли вивідна статистика дозволяє робити висновки про генеральну сукупність на основі вибірки. Впевненість у цих висновках можна представити чисельно.

\item  Дайте означення генеральної сукупності і вибірки (через випадкові величини).

Генеральна сукупність — це множина всіх значень, яких
може набувати дана випадкова величина

$\{x_1, \ldots , x_n\}$ – це вибірка спостережень за випадковою
величиною $\xi$ із генеральної сукупності (набір з
незалежних і однаково розподілених випадкових величин ("копій"). Математичне сподівання кожної вибірки дорівнює середньому генеральної сукупності.

\item  Яка вибірка вважається репрезентативною?

У якої основні характеристики вибірки (зокрема середнє) та генеративної сукупності збігаються. 

\item  Яка оцінка вибірки називається незміщенною? 

Математичне сподівання середнього вибірки дорівнює середньому генеральної сукупності: \[E(\bar{x})= \mu\] 

\item  Яка оцінка вибірки називається конзистентною?

Математичне сподівання дисперсії вибірки дорівнює $\frac{n-1}{n}$ дисперсії генеральної сукупності: \[E(s^2) = \frac{n-1}{n} \sigma^2\] 

\item  Для доведення конзистентності вибіркової дисперсії ми використовуємо закон великих чисел у формі Чебишова чи Бернуллі?

Так

\item  Як знайти стандартну похибку середнього (Standart Error)?
\[SE=\sqrt{\frac{s^2}{n}}\]

\item  Як знайти розмір вибірки для оцінювання середнього для певного рівня довіри?
\[n=(\frac{t\cdot s}{\Delta})^2 = (\frac{t\cdot \sqrt{pq}}{\Delta})^2   \; ,\] 

де t - квантиль порядку P
\[  t = \frac{\bar{x}-\mu}{\sigma} = \frac{\Delta\sqrt{n}}{s} \]

\item  Як знайти довірчий інтервал для оцінювання пропорції.

\[ p \pm Z_{k\%} \cdot SE(\bar{p}) \;, \] де k = 90\%, 95\%, 99\% \ldots
\[SE(\bar{p}) = \sqrt{\frac{p(1-p)}{n}}\]

\item  Що означає фраза “рівень довіри в 95\%” для певного довірчого інтервалу?

На 95\% можемо бути впевнені, що наше справжнє значення міститься в цьому інтервалі

\end{enumerate}
\end{document}