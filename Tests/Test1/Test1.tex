\documentclass[12pt]{extarticle} 
\usepackage[T1,T2A]{fontenc}
\usepackage[utf8]{inputenc}
\usepackage[top=2.5cm,bottom=2.5cm,left=2.5cm,right=2.5cm]{geometry}
\usepackage[english,ukrainian]{babel}
\usepackage{indentfirst} % indents first par

\usepackage{graphicx}
\graphicspath{ {./images/} } % for inserting images
\usepackage{float} %\begin{figure}[H] to force image placement
\usepackage{amsfonts,amsmath,amsthm,amscd,amssymb,latexsym, mathtools}

\title{Data Analysis \\  Тест до лекції 1}
\author{Бойченко Вікторія}
\date{}

\begin{document}

\maketitle

\begin{enumerate}

\item  Чому необхідно мати декілька характеристик центральної тенденції? Яка з них стійка до викідів?

\textbf{Бо може виникнути ситуація, коли викид сильно впливає на обрану тенденцію. Тобто, на прикладі обчислення середнього віку групи з вибіркою 18, 20, 19, 20, 23... при введені одного неправильного значення 2002 (рік замість віку) середнє значення буде аномальним, коли інші характеристики - медіана чи мода - вкажуть на спотворення значення та наявність викиду.}

\textbf{Стійка до викидів - медіана}

Мода може бути теж нестійкою

\item Яка характеристика розсіювання застосовується до вибіркового середнього?

\textbf{Дисперсію та середньоквадратичне відхилення}

\item Яка характеристика розсіювання дозволяє оцінити розкид 50${\%}$ вибірки і не враховує вплив викідів?

\textbf{Міжквартильний розмах}

\item Якщо розподіл має довгий правий хвіст, то коефіцієнт асиметріїї додатній чи від’ємний?

\textbf{Додатній}

\item Якщо коефіцієнт асиметрії від’ємний, товибіркове середнє менше за медіану. Це твердження справедливе?

\textbf{Так}

\item Якщо розподіл плосковершинний, то яким має бути коефіцієнт ексцесу? А якщо гостровершинний?

\textbf{Для плосковершинного коефіцієнт ексцесу < 3, для гостровершинного > 3}

\item Як і для чого здійснюється z-стандартизація даних?

\textbf{За формулою зводимо до нормального розподілу $\mathcal{N}\sim(0,1)$ } 

\[ z=\frac{x-\mu}{\sigma} \]

\textbf{де $\mu$ - середнє значення, }

\textbf{$\sigma$ - стандартне відхилення; }


\textbf{Зокрема стандартизація дозволяє: }
	\begin{enumerate}
	
	\item \textbf{порівнювати нормально розподілені набори даних}
	
	\item \textbf{визначати нормальність та викиди}
	
	
	
	\end{enumerate}

\end{enumerate}
\end{document}